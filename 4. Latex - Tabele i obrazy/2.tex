\documentclass{article}
%\usepackage[MeX]{polski}
%\usepackage[cp1250]{inputenc}
\usepackage{polski}
\usepackage[utf8]{inputenc}
\usepackage[pdftex]{hyperref}
\usepackage{makeidx}
\usepackage[tableposition=top]{caption}
\usepackage{algorithmic}
\usepackage{enumerate}
\usepackage{graphicx}
\usepackage{multirow}
\usepackage{amsmath} %pakiet matematyczny
\usepackage{amssymb} %pakiet dodatkowych symboli
\usepackage[table]{xcolor}
\usepackage{booktabs}
\usepackage{sidecap}
\usepackage{wrapfig}
\usepackage{caption}
\usepackage{subcaption}

\begin{document}


\subsection{Kolorowanie wierszy}

\begin{center}
\rowcolors{1}{green}{pink}
\begin{tabular}{lll}
xxx     & xxx   & 333 \\
xxx    & xzx  & fdf\\
yyy     & zzz   & gdf \\
yyy    & zzz  & dgd\\
\end{tabular}
\end{center}

\end{document}
